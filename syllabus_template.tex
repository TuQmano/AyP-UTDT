% This syllabus template was created by:
% Brian R. Hall
% Assistant Professor, Champlain College
% www.brianrhall.net

% Document settings
\documentclass[11pt]{article}
\usepackage[margin=1in]{geometry}
\usepackage[spanish]{babel}
\selectlanguage{spanish}
\usepackage[utf8]{inputenc}
\usepackage[pdftex]{graphicx}
\usepackage{multirow}
\usepackage{verbatim}
\usepackage{setspace}
\pagestyle{plain}
\setlength\parindent{0pt}
\usepackage{url}
\usepackage[colorlinks = true,
            linkcolor = blue,
            urlcolor  = blue,
            citecolor = blue,
            anchorcolor = blue]{hyperref}


\begin{document}

% Course information
\begin{tabular}{ l l }
  \multirow{2}{*}{\includegraphics[height=.75in,width=1.2in]{utdt.png}}  
  \\
 & \textbf{\large Maestría en Políticas Públicas} 
  \\ 
  & Escuela de Gobierno - Universidad Torcuato Di Tella\\ \\ 
  & PRIMER TRIMESTRE 2019
  
\end{tabular}
\vspace{20mm}

% Professor information
\begin{tabular}{l}
  \multirow{6}{*}  
\textbf{\LARGE Actores y Procesos en las Políticas Públicas}
 \\
  \\

\large\textbf{Martes, 19.15hs.}
 \\

\large{(12 de marzo al 21 de mayo)}
 
 \\
  \\
 \large \textbf{Juan Pablo Ruiz Nicolini}
 \\
 \large \href{mailto:juan.ruiznicolini@mail.utdt.edu}{juan.ruiznicolini@mail.utdt.edu}
\end{tabular}

%\vspace{1cm}
    %\begin{center} \textit{\textbf{[Este es un programa tentativo sujeto a modificaciones]}} \\
%\end{center}

\vspace{2cm}

% Course details \footnote{El programa del curso está basado en la estructura de la materia dictada en otras ocasiones por Germán Lodola}
\textbf {\large Descripción} \\ 

El presente curso ofrece una serie de elementos teóricos y prácticos para la comprensión del proceso de formación de las políticas públicas. Con este objetivo, el curso discute los factores que permiten establecer cuál es la naturaleza de los problemas públicos, qué mecanismos y actores influyen sobre la incorporación de los problemas públicos a la agenda de los gobiernos, cómo se lleva a cabo la búsqueda de soluciones y se arriba finalmente a una decisión de política, y cuáles son los rasgos salientes de los procesos de implementación y evaluación de políticas. A través del examen del ciclo de las políticas públicas esbozado anteriormente, el curso identifica y analiza la influencia ejercida por las estructuras burocráticas, los partidos políticas, los grupos de interés, los medios de comunicación y las instituciones de gobierno (presidentes y legislaturas). 

\vspace{.3cm}

\textbf {\large Organización}  \\

El curso se desarrollará durante once semanas. Las sesiones consistirán en una presentación general de los temas de cada semana y una  discusión de la bibliografía seleccionada. Se propiciará también poder discutir los contenidos del programa ilustrándolo con ejemplos concretos de la actualidad política nacional y (cuando fuera pertinente) de otros países de la región y el mundo. 

\vspace{.3cm}

\textbf {\large Evaluación} \\

El sistema de evaluación del curso supone la realización de un examen escrito con fecha inamovible. Se espera, además, que los estudiantes concurran a clase con el material de lectura debidamente preparado. La calidad y frecuencia de las intervenciones será oportunamente evaluada.
\clearpage


\textbf {\large Cronograma y lecturas} \\ A continuación, se lista el material de lectura obligatoria. Toda la literatura se encuentra disponible en formato electrónico en el campus virtual de la universidad (\href{http://campus.utdt.edu/}{http://campus.utdt.edu/}) y en ésta carpeta compartida de \href{https://www.dropbox.com/sh/8vwheg39c0vzuj4/AACnMI5McG29M_7vontcb0rJa?dl=0}{Dropbox}.

\vspace{.5cm}

\textbf{Semana I:}\textbf{ PRESENTACIÓN}
\vspace{.5cm}

\textbf{Semana II} - \textbf{Enfoques para el estudio de las políticas públicas: la economía política de las políticas públicas. El ciclo de las políticas públicas. Las dimensiones políticas y
técnicas de la elaboración de políticas. Tipos de políticas públicas.}

\begin{onehalfspacing}
\hangindent=.25in \noindent \textbf{(1)} Offe, Claus. 2001. “Economía política: Perspectivas sociológicas”. En Robert E. Goodin y Hans-Dieter Klingemann (eds.) Nuevo Manual de Ciencia Política, Madrid, Istmo, pp. 981-1002.

\hangindent=.25in \noindent \textbf{(2)} Lowi, Theodore. 1972. “Four Systems of Policy, Politics, and Choice.” Public Administration
Review 32 (4): 298-310.

\hangindent=.25in \noindent \textbf{(3)} Birkland, Thomas. 2001. An Introduction to the Policy Process. M.E. Sharpe. Capítulo 6
\end{onehalfspacing}

\vspace{.5cm}
\textbf{Semana III} - \textbf{El concepto de problema público. Identificación de las causas y definición del problema.}
\begin{onehalfspacing}

\hangindent=.25in \noindent\textbf{(4)} Aguilar Villanueva, Luis. 1993. Problemas Públicos y Agenda de Gobierno. México: Miguel Angel Porrúa Editor, pp. 51-72.

\hangindent=.25in \noindent \textbf{(5)} Stone, Deborah. 1989. “Causal Stories and the Formation of Policy Agendas”. Political Science Quarterly 104 (2): 281-300.
\end{onehalfspacing}

\vspace{.5cm}
\textbf{Semana IV} - \textbf{La formación de la agenda pública. Incorporación de un problema a la agenda. Formulación de alternativas y toma de decisiones. Tipos de agenda: agenda pública y agenda de gobierno.}

\begin{onehalfspacing}
\hangindent=.25in \noindent \textbf{(6)} Kingdon, John. 1995. “Agenda Setting”. En Steve Theodoulou and Mattew Cahn (eds.) Public Policy. The Essential Readings. Prentice Hall, Capítulo 13

\hangindent=.25in \noindent \textbf{(7)} Cobb, Roger W., Jeannie Keith-Ross, and Marc Howard Ross. 1976. “Agenda Building as a Comparative Political Process.” American Political Science Review 70: 126–38
\end{onehalfspacing}.
\vspace{.5cm}

\textbf{Semana V} - \textbf{El control de la agenda: Usos del poder.}

\begin{onehalfspacing}

\hangindent=.25in \noindent \textbf{(8)} Lukes, Steven. 1985. El poder. Un estudio radical. Madrid: Siglo XXI.

\hangindent=.25in \noindent \textbf{\textbf{(*)}} Auyero, Javier y Débora Swistun. 2006. “Tiresias en Villa Inflamable. Hacia una cronografía de la dominación”. Cuadernos del CISH, (19-20).
\end{onehalfspacing}

\vspace{.5cm}
\textbf{Semana VI} - \textbf{La dinámica de atención pública y la influencia de los medios de comunicación.}

\begin{onehalfspacing}
\hangindent=.25in \noindent \textbf{\textbf{(9)}} Curran, James. 2002. “Media and democracy: The third way”. En James Curran Media and Power, Capítulo 8, 217-260.

\hangindent=.25in \noindent \textbf{\textbf{(10)}} Lodola, Germán y Philip Kitzberger. 2017. "Politización y confianza en los medios de comunicación: Argentina durante el kirchnerismo." Revista Ciencia Política 37 (3): 635-658.
\end{onehalfspacing}

\clearpage
\vspace{.5cm}
\textbf{Semana VII} - \textbf{El papel de las instituciones. Las instituciones de gobierno: Presidentes y Legislaturas}

\begin{onehalfspacing}
\hangindent=.25in \noindent \textbf{(*)} Acuña, Carlos y Mariana Chudnovsky.2013. “Cómo entender las instituciones y su relación con la Política”. En Carlos Acuña (ed) ¿Cuánto importan las instituciones? Buenos Aires: Siglo XXI. Pp19-39 / 61-64

\hangindent=.25in \noindent \textbf{(11)} Cox, Gary y Scott Morgenstern. 2001. “Legislaturas reactivas y presidencias proactivas en América Latina.” Desarrollo Económico 41 (163): 373-93.

\hangindent=.25in \noindent \textbf{(12)} Bonvecchi, Alejandro y Javier Zelaznik. 2010. Recursos de Gobierno y Funcionamiento del Presidencialismo en Argentina, Mimeo UTDT.
\end{onehalfspacing}

\vspace{.5cm}

\textbf{Semana VIII} - \textbf{Las políticas públicas en los sistemas federales de gobierno.}
\begin{onehalfspacing}

\hangindent=.25in \noindent \textbf{(13)} Stepan. Alfred. 2004. “Electorally Generated Veto Players in Unitary and federal Systems.” In Edward Gibson ed., Federalism: Latin America in Comparative Perspective. Baltimore: Johns Hopkins University Press.


\hangindent=.25in \noindent \textbf{(14)}  Bonvecchi, Alejandro y Germán Lodola. 2011. “The Dual Logic of Intergovernmental Transfers. Presidents, Governors, and the Politics of Coalition-Building in Argentina”. Publius: The Journal of Federalism 41 (2): 179-206.

\hangindent=.25in \noindent \textbf{(*)}  Niedzwiecki, Sara. 2016. "Social Policies, Attribution of Responsibility, and Political Alignments. A Subnational Analysis of Argentina and Brazil". Comparative Political Studies 49 (4), pp. 457-498.
\end{onehalfspacing}

\vspace{.5cm}
\textbf{Semana IX } - \textbf{La decisión de las políticas públicas. El papel del contexto, las ideas y las comunidades epistémicas.}

 \hangindent=.25in \noindent\textbf{(15)} Keeler, John. 1993. “Opening the Window for Reform. Mandates. Crises, and Extraordinary Policymaking.” Comparative Political Studies 25 (2): 435-86.

\hangindent=.25in \noindent\textbf{(16)} Campbell, John. 2002. “Ideas, Politics, and Public Policy”. Annual Review of Sociology 28: 21- 38.

 \hangindent=.25in \noindent\textbf{(17)} Weir, Margaret y Theda Skocpol. 1993. “La estructura del estado: una respuesta Keynesiana a la Gran Depresión en Suecia, Gran Bretaña y Estados Unidos.” Zona Abierta 63/64: 73-153.
 
\vspace{.5cm}
\textbf{Semana X } - \textbf{Políticas Públicas en el presidencialismo argentino}.

\hangindent=.25in \noindent \textbf{(18)} Spiller, Pablo y Mariano Tommasi. 2011. “Un país sin rumbo: ¿Cómo se hacen las políticas públicas en Argentina”. en Scartascini, Carlos et al. (eds.): El Juego Político en América Latina: ¿Cómo se Deciden las Políticas Públicas?, Washington DC, Banco Interamericano de Desarrollo: 75-116.

\textbf{Semana XI: EXAMEN FINAL} (21 de mayo)  

\vspace{2cm}
(Los textos marcados con \textbf{*} son sugeridos). 


\end{document}



